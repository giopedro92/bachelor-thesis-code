%-----------------------------CAPITOLO 1
\section{Il Modello Standard (SM)}
\label{sec:SM}
    La fisica delle particelle elementari ha lo scopo di indagare la struttura microscopica della materia andando alla ricerca dei suoi costituenti ultimi e delle loro interazioni. L'insieme delle teorie che meglio hanno saputo descrivere le evidenze sperimentali ha trovato una coerente formulazione teorica nel Modello Standard (Standard Model, MS) della fisica delle particelle elementari. Ad oggi il modello prevede l'esistenza di tre tipologie di particelle elementari: quark, leptoni e bosoni mediatori i quali rappresentano tre delle quattro interazioni fondamentali, come rappresentato in figura~\ref{fig:1-standard-model}, esclusa quella gravitazionale non spiegabile con le teorie attuali.
    
    Il Modello Standard descrive dodici campi materiali dotati di massa che rappresentano i dodici sapori delle particelle materiali classificate in base alle loro interazioni. Queste particelle di spin \sfrac{1}{2} sono dette \textit{fermioni} poiché seguono la statistica di Fermi-Dirac. I fermioni si dividono in sei quark e sei leptoni: i primi sono soggetti a tutte le interazioni naturali, mentre i secondi non interagiscono con la forza forte~\cite{CG_2007}.
    
    I \textit{quark} up e down, charm e strange, top e bottom sono organizzati in doppietti o generazioni nelle quali il primo elemento è generalmente il più massivo e ha carica elettrica positiva di modulo uguale a \sfrac{2}{3} quella dell'elettrone, mentre il secondo ha carica elettrica negativa di modulo uguale a \sfrac{1}{3} di quella dell'elettrone.

    I \textit{leptoni} il cui nome deriva dal greco \textit{leptos}, leggero, poiché solitamente di massa inferiore ai quark, sono organizzati in doppietti: elettrone, muone e tauone e relativi neutrini elettronico, muonico e tauonico; i primi tre hanno carica elettrica negativa e unitaria, mentre i neutrini hanno carica elettrica e massa nulle secondo il Modello Standard. Le più recenti evidenze sperimentali mostrano però che i neutrini acquisiscono massa attraverso meccanismi ancora ignoti.

    Alle dodici particelle elementari corrispondono dodici \textit{antiparticelle} teorizzate per la prima volta nel 1929 dal fisico britannico Paul Dirac. Queste particelle hanno caratteristiche fisiche come massa, spin e vita media uguali a quelle delle relative particelle, ma numeri quantici e cariche opposte.

    \begin{figure}[t]
        \centering
        \includegraphics[width=0.8\linewidth]{res/fig/1-chapter/1-Standard_Model_of_Elementary_Particles.pdf}
        \caption{Schema delle particelle elementari presenti nel Modello Standard con relativa massa, carica e spin~\cite{Wikimedia_Standard_Model}.}
        \label{fig:1-standard-model}
    \end{figure}

    In seguito troviamo le particelle mediatrici delle interazioni fondamentali. Queste particelle hanno spin 1 e sono dette \textit{bosoni}, vettore o di gauge, poiché seguono la statistica di Bose-Einstein e corrispondono alle tre interazioni fondamentali spiegate dal Modello Standard: otto gluoni $g$ mediatori dell'interazione forte, ciascuno con tre cariche di colore possibili, il fotone $\gamma$ mediatore dell'interazione elettromagnetica e i bosoni $Z^0$ e $W^\pm$ mediatori dell'interazione debole.

    Infine nel 2012 al CERN di Ginevra è stato scoperto dagli esperimenti ATLAS~\cite{ATLAS_2012} e CMS~\cite{CMS_2012} un bosone scalare di spin \num{0} chiamato bosone di Higgs $H$ associato al campo di Higgs col quale interagiscono tutte le particelle massive, fermioniche o bosoniche, per ottenere la loro massa tramite un meccanismo detto di \textit{rottura spontanea della simmetria} ipotizzato nel 1964 da F. Englert e R. Brout~\cite{Englert_1964}, Peter W. Higgs~\cite{Higgs_1964} e G. S. Guralnik, C. R. Hagen e T. W. B. Kibble~\cite{GHK_1964}.
    
    È bene precisare che quelle presentate non sono particelle in senso classico, ma si fa sempre riferimento a campi quantizzati in cui i campi materiali possiedono cariche interne che permettono l'accoppiamento coi relativi campi di forza. Le teorie che compongono il Modello Standard sono teorie di campo quantizzato (Quantum Field Theory, QFT): la \textit{Teoria Elettrodebole} che generalizza la Elettrodinamica Quantistica (Quantum Electrodynamics, QED) e spiega i fenomeni elettromagnetici e di interazione debole e la \textit{Cromodinamica Quantistica} (Quantum Chromodynamics, QCD) che spiega l'interazione tra quark attraverso lo scambio di gluoni. Ancora però non siamo capaci di descrivere in senso quantistico l'ultima interazione naturale, quella gravitazionale, per questo non presente nel modello.

    La fisica delle particelle studia fenomeni che coinvolgono corpi di dimensioni infinitesime a velocità prossime a quella della luce, è naturale quindi che il formalismo matematico del Modello Standard sia quello delle teorie di campo quantizzato che rappresentano l'evoluzione della meccanica quantistica in ambito relativistico e permettono lo studio di fenomeni sia quantistici sia relativistici e la creazione e distruzione di particelle.
    
    Il concetto di campo quantizzato è associato sia alle particelle sia alle loro interazioni: le prime sono interpretate come manifestazione del relativo campo, le ultime come scambio di quanti virtuali col campo di forza relativo all'interazione in gioco. Il Modello Standard è una teoria quantistica di campo di gauge locale che nel linguaggio della teoria dei gruppi di simmetrie si indica come $SU(3)_C \times SU(2)_L \times U(1)_Y$ in cui, da sinistra, sono racchiuse le tre interazioni naturali: forte, debole e elettromagnetica.

    Le interazioni nucleari forti sono a corto raggio per cui confinate all'interno degli adroni e descritte dalla simmetria inviolata $SU(3)_C$, dove $C$ sta per colore, sulla quale poggia la \textit{Cromodinamica Quantistica} (QCD, vedi la sezione~\ref{sec:QCD}). Lo spazio di $SU(3)_C$ ha $3^2 - 1 = 8$ generatori, cioè otto bosoni di gauge di spin 1 chiamati \textit{gluoni} $g$ mediatori dell'interazione forte. Come detto prima, tra i fermioni solo i quark sono soggetti all'interazione forte ovvero possiedono la carica, di colore, di $SU(3)_C$ che può assumere tre valori convenzionalmente indicati come $red\ (r)$, $green\ (g)$ e $blue\ (b)$ e rispettivi anticolori. I quark interagiscono tra loro scambiando gluoni dotati di una doppia carica di colore: colore-anticolore, a differenza dei fotoni che sono elettricamente neutri e cioè non possiedono la carica del campo che mediano. Per questo motivo i gluoni possono interagire tra loro, mentre i fotoni no. Matematicamente questa differenza è descritta dalla non abelianità del gruppo della QCD e dalla abelianità di quello della QED.

    Le interazioni debole e elettromagnetica sono descritte e unificate dalla simmetria $SU(2)_L \times U(1)_Y$, dove $L$ sta per leptoni e $Y$ per hypercharge, ipercarica, e dal meccanismo di Higgs di rottura della simmetria che permette alle particelle di acquisire la loro massa. Nella forma più semplice questo meccanismo produce 4 bosoni di gauge, vettori, di spin~\num{1}: due neutri di cui uno massivo ($Z^0$) e uno privo di massa ($\gamma$) e due carichi e massivi ($W^\pm$) più un bosone scalare di spin \num{0}, il bosone di Higgs ($H$)~\cite{Vitale_1995}.

\newpage

\section{Adroni e Modello a Quark (QPM)}
\label{sec:QPM}
    I dati relativi agli esperimenti di diffusione profondamente inelastica $ep$ suggerirono un modello fenomenologico dell'interno dell'adrone che prese il nome di \textit{modello a partoni}. Proposto da Richard Feynman nel 1969, questo modello ipotizza che i nucleoni, costituenti del nucleo atomico, non siano particelle elementari, ma siano costituiti da centri diffusori puntiformi detti partoni. In seguito i partoni vennero identificati con quark e gluoni e oggigiorno il termine \textit{partone} indica quark e gluoni costituenti di un adrone indifferentemente.
    
    Col termine \textit{adrone} indichiamo le particelle composte di quark, solitamente più pesanti dei leptoni, il cui nome deriva dal greco \textit{hadrón}, pesante, che possiedono carica di colore e che possono quindi interagire tramite forza forte. Solitamente i quark che costituiscono l'adrone vengono chiamati \textit{quark di valenza}, mentre gluoni, quark e antiquark virtuali generati dalle forze forti che uniscono i quark di valenza vengono chiamati \textit{mare}.

    Il Modello a Quark noto anche come Modello a Partoni o Modello Quark-Partone (Quark-Parton Model, QPM), è un modello che descrive gli adroni come composti di quark fornendone una semplice classificazione. Poiché i quark liberi, ovvero non legati assieme all'interno di un adrone, non sono mai stati osservati, è stato \textit{postulato} che i quark siano confinati all'interno degli adroni, come verrà chiarito meglio nella sezione~\ref{sec:QCD}.

\section{La Cromodinamica Quantistica (QCD)}
\label{sec:QCD}
    Chiarita la struttura interna degli adroni, la QCD ci fornirà ora un quadro teorico più completo per descrivere le interazioni tra quark e gluoni.

    Come già accennato nella sezione~\ref{sec:SM}, la \textit{Cromodinamica Quantistica} (QCD) è la teoria di campo quantizzato che descrive l'interazione forte attraverso scambi di gluoni. È una teoria di gauge non abeliana con gruppo di simmetria $SU(3)_C$, possiede quindi 8 generatori o bosoni di gauge vettori di spin 1 mediatori dell'interazione forte, chiamati \textit{gluoni} $g$ che possiedono a loro volta una doppia carica di colore: colore-anticolore. La QCD mostra come le uniche combinazioni di quark possibili per formare un adrone siano \textit{mesoni} (coppie quark-antiquark) e \textit{barioni} (tripletti di quark e antiquark). Nonostante questo sono stati sperimentalmente osservati stati esotici di quattro e cinque quark e stati legati di soli gluoni.

    Sperimentalmente non sono mai stati osservati quark liberi a causa del cosiddetto \textit{confinamento di colore}: i quark si legano in doppietti o tripletti che devono necessariamente essere di colore bianco ovvero neutri cioè con carica di colore nulla. Il confinamento di colore prevede infatti che sia energeticamente favorevole la produzione di una ulteriore coppia quark-antiquark, chiamata \textit{jet adronico}, nel caso si tentasse la separazione tra quark e antiquark in un mesone fornendo energia, rendendo impossibile l'ottenimento di un quark libero come mostrato in figura~\ref{fig:2-hadron-jet}.

    \begin{figure}[t]
        \centering
        \includesvg[width=1\linewidth]{1-chapter/2-Quark_confinement}
        \caption{Rappresentazione grafica della rottura di stringa QCD nel vuoto~\cite{Wikimedia_Quark_Confinement}. La figura mostra come venga generata una coppia quark-antiquark quando un mesone riceve energia sufficiente: il gluone che lega i due quark si ``allunga'' finché non si spezza e forma una nuova coppia quark-antiquark.}
        \label{fig:2-hadron-jet}
    \end{figure}

    Un'altra importante proprietà della QCD è la \textit{libertà asintotica} secondo la quale l'intensità dell'interazione forte è estremamente bassa ad alte scale di energia o piccole distanze; questo comporta che a brevissima distanza i quark siano sostanzialmente liberi. In regime di alte energie o piccole distanze invece l'interazione è molto meno intensa permettendo l'utilizzo di approcci di calcolo \textit{perturbativi}~\cite{BGS_2012}.

    \subsection{Cromodinamica Quantistica Perturbativa (pQCD)}
        Due sono gli approcci tradizionali alla QCD perturbativa. Il primo è il metodo detto dell'\textit{Elemento di Matrice} (Matrix Element, ME)~\cite{Vitale_1995} in cui i diagrammi di Feynman sono calcolati compiutamente per ogni ordine. In linea di principio questo metodo è il più rigoroso, ma incontra grandi difficoltà già al terzo ordine, tanto che i soli calcoli ad ora disponibili si arrestano al secondo ordine perturbativo.

        Il secondo approccio è quello della Cascata di Partoni (\textit{Parton Shower}, PS)~\cite{Bambah_1989}. Si tratta in questo caso di produrre un numero arbitrario di partoni che combinati tra loro generano gli eventi a più jet. Questo è possibile poiché non vengono utilizzate le espressioni complete degli elementi di matrice, ma solo delle loro approssimazioni.

        Per determinare il regime in cui la teoria perturbativa è applicabile è necessario valutare il valore di $Q^2 = - q^2$ con $q$ \textit{quadrimomento} trasferito nella collisione e segno negativo derivante dalla metrica di Minkowski~\cite{Altarelli_2004}. Nel nostro caso, per grandi valori di $Q^2$, l'interazione forte diventa meno intensa e può quindi essere trattata con metodi perturbativi rendendo così la \textit{QCD perturbativa} (perturbative QCD, pQCD) un approccio valido.

\newpage

\section{Plasma di Quark e Gluoni (QGP)}
\label{sec:QGP}
    Un modello euristico che permette di descrivere i quark confinati negli adroni è il \textit{MIT bag model}~\cite{Wong_1994}. Secondo questo modello i quark sono particelle di massa nulla all'interno di una scatola di dimensioni finite e infinitamente massivi all'esterno. In questo modo il confinamento non è altro che il risultato del bilancio tra pressione esterna e interna, quest'ultima data dall'energia cinetica dei quark stessi. I gluoni scambiati tra quark sono anch'essi confinati nella scatola la cui carica di colore totale deve essere nulla.
    
    Questo modello fornisce ragioni sufficienti del perché ci aspettiamo di trovare nuove fasi della materia formata da quark oltre alla materia adronica: se la pressione esercitata dai quark interni crescesse oltre il valore della pressione esterna si verrebbe a creare un nuovo stato della materia a temperature e pressioni altissime in cui quark e gluoni non sono più legati, chiamato \textit{Plasma di Quark e Gluoni} (Quark Gluon Plasma, QGP). Una rappresentazione indicativa del diagramma di fase della materia fortemente interagente è riportato in figura~\ref{fig:3-phase-diag-qgp}.
    
    \begin{figure}[h]
        \centering
        \includegraphics[width=0.7\linewidth]{res/fig/1-chapter/3-PhasDiagQGP.png}
        \caption{Diagramma di fase qualitativo della materia fortemente interagente~\cite{Wikipedia_PhasDiagQGP}.}
        \label{fig:3-phase-diag-qgp}
    \end{figure}
    
    Quando il sistema raggiunge la temperatura critica $T_C \approx$ \qtyrange[range-phrase = --, range-units = single]{150}{200}{\mega \eV} avviene una transizione di fase del primo ordine tra materia adronica e QGP. La densità di energia nell'intorno di questa transizione presenta una discontinuità detta \textit{calore latente di deconfinamento}. La regione di basse temperature e alte densità è detta regione di \textit{diquark matter} o regione di \textit{superconduttività di calore}. In queste condizioni avviene la formazione di coppie di quark non neutre di colore analoghe alle coppie di Cooper dei superconduttori.

    La capacità dei barioni di ricombinare i quark per formare nuovi adroni è chiamata potenziale chimico barionico o \textit{potenziale bariochimico}, influenza la composizione di barioni e mesoni prodotti nelle collisioni ed è formulato come segue:
    
    \begin{equation}
        \mu_B = \dv{E}{N_B}.
    \end{equation}

    Un caso particolare è il QGP fortemente legato (\textit{strongly-coupled QGP}, sQGP) nel quale le interazioni tra quark e gluoni sono estremamente forti e non possono essere trattate con gli approcci convenzionali basati sulla teoria perturbativa.

\section{Collisioni tra particelle}
\label{sec:COLLISION}
    Nello studio della QCD le predizioni teoriche si basano sul calcolo perturbativo di sistemi di quark e gluoni dotati di colore, carica che rappresenta i gradi di libertà a piccola distanza; le osservazioni sperimentali invece sono fatte su stati finali di adroni, ovvero stati legati di partoni in singoletto di colore. È quindi necessario analizzare i processi che portano allo stato finale cercando di descriverli con le caratteristiche dell'interazione partonica iniziale.

    Le collisioni tra ioni pesanti (A-A), ovvero tra i nuclei dei due atomi, accelerati ad energie relativistiche sono lo strumento migliore per studiare la materia nucleare in condizioni estreme di temperatura e densità di energia. È così possibile produrre numerose collisioni simultanee tra i vari nucleoni presenti nei due nuclei collidenti, creando un sistema ad altissima densità di partoni interagenti tra loro e riproducendo così in laboratorio le condizioni dell'universo primordiale frazioni di secondo dopo il Big-Bang, quando si presume che la materia in espansione si presentasse sotto forma di QGP.

    L'avvenuta formazione di uno stato di QGP può essere verificata attraverso la misura di diversi effetti come gli spettri di impulso delle particelle prodotte, la soppressione o l'aumento di produzione di stati legati di quark pesanti e la presenza di moti collettivi. È poi necessario confrontare le misure ottenute con quelle di collisioni protone-protone ($pp$) alle stesse energie per assicurarsi che lo stato prodotto in collisioni A-A non sia una semplice sovrapposizione di urti $pp$ e che si tratti effettivamente di QGP e non di un gas di adroni eccezionalmente denso.

    A complicare ulteriormente questo quadro, le differenze tra i risultati ottenuti in collisioni A-A e $pp$ potrebbero essere dovute all'utilizzo di proiettili estesi, i nuclei, nel primo caso, in particolare per le modifiche alle funzioni di distribuzione partonica nei nucleoni appartenenti ad un nucleo e alla presenza di scattering multipli prima di un hard scattering. È quindi necessario accertarsi che i risultati ottenuti in collisioni A-A non dipendano da questi effetti denominati di \textit{effetti di stato iniziale}. Per fare ciò vengono utilizzate collisioni protone-ione $p$-A.

    Negli anni sono state ottenute numerose evidenze sperimentali a favore della formazione del QGP in collisioni A-A da esperimenti ai collider SPS~\cite{NA60_2021}, RHIC~\cite{Gyulassy_2004_RHIC_QGP} e LHC~\cite{ALICE_2024}. Recentemente però in collisioni $pp$ e $p$-A sono stati \textit{osservati} gli stessi effetti normalmente associati al deconfinamento di quark, ovvero al QGP, e per questo motivo assolutamente non attesi: aumento di produzione di particelle strange, evidenze di collettività a basso impulso trasverso e presenza di correlazioni a lungo raggio con conseguenti misure di flusso ellittico e armoniche superiori. L'esperimento ALICE in particolare ha osservato un aumento della stranezza, ovvero della produzione di particelle contenenti al loro interno uno o più quark strange $s$, in funzione della molteplicità dell'evento~\cite{ALICE_2008} confrontando collisioni $pp$, $p$-A e A-A in cui eventi $pp$ ad alta molteplicità mostrano risultati molto simili a quelli ottenuti in collisioni A-A~\cite{RHIC_2020}~\cite{ALICE_2017_pp}~\cite{ALICE_2024_pp_pPb_PbPb}. Una possibile spiegazione di questo fenomeno consiste nell'assumere che anche in collisioni $pp$ ad alta molteplicità, in cui cioè avvenga più di una collisione partone-partone tra i costituenti dei due protoni interagenti, si possa creare uno stato di QGP. Questa ipotesi può spiegare la presenza di effetti collettivi negli stati finali. D'altro canto, le ridotte dimensioni del QGP eventualmente creato in collisioni $pp$ e $p$-A sarebbero in accordo con la \textit{mancata osservazione} di fenomeni di perdita di energia per partoni ad alto impulso trasverso nell'attraversare un mezzo con elevata densità di partoni liberi.

    \subsection{Collisioni tra ioni pesanti}
        L'unico metodo conosciuto per creare condizioni di temperatura e densità energetica così elevate da produrre artificialmente il QGP in laboratorio sono le collisioni tra ioni pesanti ultrarelativistici, $\beta = v/c \approx 1$.

        Le dimensioni dei nuclei degli ioni collidenti sono molto maggiori rispetto a tutte le scale proprie della fisica delle particelle elementari che appunto studia i quark costituenti dei nucleoni che a loro volta costituiscono il nucleo. Per questa ragione la \textit{geometria delle collisioni} gioca un ruolo fondamentale nell'analisi e interpretazione dei risultati sperimentali.

        Nel sistema del centro di massa, grazie alla contrazione di Lorentz nella direzione longitudinale di propagazione del fascio, i due nuclei possono essere visti nel piano trasverso come dischi sottili di raggio $2 R_A \approx 2 A^{\frac{1}{3}}$ con $A$ numero di nucleoni. Una rappresentazione grafica del processo è rappresentata in figura~\ref{fig:4-centrality}; alcune delle quantità rilevanti sono~\cite{Salgado_2009}:
        \begin{description}
            \item[\textit{parametro di impatto} $b$] distanza tra gli assi centrali dei nuclei in procinto di collidere, che caratterizza la centralità della collisione: l'urto si dirà centrale se $b$ è molto piccolo e lo scontro è pressoché frontale, si dirà invece periferico se $b$ è grande rispetto alle dimensioni delle particelle. La centralità dell'evento si esprime tipicamente in percentuali di sezione d'urto totale.
            
            \item[numero di nucleoni coinvolti] i \textit{participants}, $N_{part}$ all'interno dei nuclei collidenti ossia il numero di neutroni e protoni dei due ioni che prendono parte alla collisione. I restanti vengono chiamati spettatori, \textit{spectators}, e proseguono nella loro traiettoria quasi imperturbati. 
            
            \item[numero totale di collisioni nucleone-nucleone incoerenti] $N_{coll}$.
        \end{description}
        Numerosi modelli teorici sono stati sviluppati per descrivere le dinamiche di collisione a partire da queste quantità.

        \begin{figure}[t]
            \centering
            \includegraphics[width=1\linewidth]{res/fig/1-chapter/4-centrality.png}
            \caption{Rappresentazione geometrica della collisione tra i nuclei di due ioni pesanti~\cite{Toia_2013}. Come mostrato, la collisione può coinvolgere solamente parte dei nucleoni presenti (\textit{participants}), lasciandone fuori altri (\textit{spectators}).}
            \label{fig:4-centrality}
        \end{figure}

         In fisica subnucleare si è soliti descrivere le traiettorie delle particelle in termini della variabile \textit{rapidità} $y$ o della variabile \textit{pseudorapidità} $\eta$, definite rispettivamente come:
         \begin{equation*}
             y = \frac{1}{2} \ln(\frac{E + p_L}{E - p_L})
         \end{equation*}
        con $E = \sqrt{p^2 c^2 + m^2 c^4}$ energia relativistica e $p_L$ componente longitudinale del momento della particella rispetto all'asse del fascio e:
        \begin{equation*}
            \eta = \frac{1}{2} \ln(\frac{p + p_L}{p - p_L}) = - \ln(\tan{\frac{\theta}{2}})
        \end{equation*}
        con $\theta$ angolo tra impulso della particella e asse del fascio. Per particelle di massa nulla come il fotone rapidità e pseudorapidità coincidono, mentre per particelle massive questa corrispondenza vale solo nel limite ultrarelativistico.

\newpage

\section{Evoluzione del QGP}
    Avvenuta la collisione si ha la formazione di un plasma di quark e gluoni solo nel caso in cui vengano raggiunte le condizioni critiche di temperatura e densità di energia~\cite{Andronic_2014}.
    
    Nel caso in cui le condizioni richieste non venissero raggiunte il sistema entrerebbe in \textit{evoluzione idrodinamica}, figura~\ref{fig:5-time-evol-qgp} a) a sinistra. Questo è il caso tipico delle collisioni tra protoni $pp$ o di collisioni tra ioni pesanti A-A non sufficientemente energetiche e centrali. Subito dopo la collisione vi è una breve fase \textit{pre-adronica}, in grigio, in cui i quark prodotti adronizzano in un ``vuoto'' QCD dopo la quale il sistema evolve come \textit{gas di adroni}, in sostanza attraverso processi di frammentazione. Sebbene avvenga un sostanziale incremento di pressione e temperatura non si manifesta alcun deconfinamento di partoni in questo caso.

    Nel caso in cui invece la collisione fosse sufficientemente energetica da soddisfare le condizioni di creazione del QGP, il processo rappresentato in figura~\ref{fig:5-time-evol-qgp} b) a destra è più complesso~\cite{Strazzi_2019}:
    \begin{enumerate}
        \item \textit{Pre-Equilibrium phase} ($t < \tau_0 \approx$ \qty[per-mode = symbol]{1}{\femto \meter \per \clight}): in questa fase i partoni diffondono l'uno sull'altro producendo quark e gluoni deconfinati in abbondanza. Vengono prodotte molte particelle ad elevato impulso trasverso ($p_T \gg$ \qty[per-mode = symbol]{1}{\giga \eV \per \clight}) e una grande quantità di fotoni sia reali sia virtuali che decadono in coppie leptone-antileptone.

        \item \textit{Termalizzazione} ($t \approx$ \qtyrange[range-phrase = --, range-units = single, per-mode = symbol]{1}{10}{\femto \meter \per \clight}): questa fase è caratterizzata dalle interazioni elastiche e inelastiche tra i partoni del QGP. Le interazioni inelastiche hanno la peculiarità di poter cambiare la composizione di sapore delle particelle. A causa della pressione interna il sistema all'equilibrio termico inizia ad espandersi rapidamente raffreddandosi di conseguenza e convertendosi in un gas adronico (\textit{fase mista}).

        \item \textit{Adronizzazione} ($t \approx$ \qty[per-mode = symbol]{20}{\femto \meter \per \clight}): durante l'espansione il sistema si raffredda raggiungendo il valore critico di densità che dà inizio al processo di adronizzazione in cui quark e gluoni del QGP condensano in nuovi adroni. L'interazione tra gli adroni continua finché il relativo tasso è in grado di sostenere l'espansione del QGP e raggiunto un certo valore della temperatura cessano le interazione inelastiche tra i costituenti del sistema. Dopodiché la composizione di sapore del QGP si fissa raggiungendo il congelamento chimico (\textit{chemical freeze-out}).

        \item \textit{Congelamento termico} (\textit{thermal freeze-out}): quando la densità del sistema è tale da rendere la distanza media tra gli adroni maggiore del raggio di azione dell'interazione forte, per $T_{fo} \approx$ \qty{120}{\mega \eV}, le diffusioni elastiche tra gli adroni cessano e resta fisso anche lo spettro cinematico della materia risultante.
    \end{enumerate}

    \begin{figure}[p]
        \centering
        \includegraphics[width=0.9\linewidth]{res/fig/1-chapter/5-TimeEvolQGP.jpg}
        \caption{Evoluzione temporale di una collisione tra ioni pesanti in un piano simile a quello di Minkowski~\cite{RS_2011}. a) a sinistra il caso in cui il sistema entra in evoluzione idrodinamica e diventa un gas adronico senza formazione di QGP. b) a destra il caso in cui le condizioni permettono la formazione del QGP che perdendo energia si trasforma anch'esso in gas adronico.}
        \label{fig:5-time-evol-qgp}
    \end{figure}

\newpage

\section{Adronizzazione di sapori pesanti in collisioni $pp$}
\label{sec:ADRONIZATIONpp}
    Nello studio delle proprietà del QGP i quark pesanti charm $c$ e bottom $b$ rivestono un ruolo fondamentale poiché in virtù della loro massa elevata vengono prodotti in collisioni \textit{hard}, ossia ad alto momento $Q^2$ trasferito, tra i partoni dei nucleoni solo nelle primissime fasi della collisione nucleo-nucleo, prima ancora che il sistema termalizzi e si formi lo stato di QGP. Questi quark quindi si propagano attraverso il sistema ultra-denso interagendo coi suoi costituenti e fornendo una \textit{misura diretta delle sue proprietà}. Per poter comprendere appieno una misura effettuata in collisioni A-A però è necessario confrontarla con la stessa misura effettuata in collisioni $pp$ e $p$-A, come chiarito nella sezione~\ref{sec:COLLISION}.

    L'adronizzazione di sapori pesanti in collisioni $pp$ attraverso il processo di \textit{frammentazione} viene descritta matematicamente attraverso il \textit{teorema di fattorizzazione}~\cite{CSS_2004}. Data la scala del momento $Q^2$ trasferito nel processo di collisione, esso consiste nel separare il contributo perturbativo ad alta energia della produzione del \textit{leading parton} dalla successiva conversione nello stato adronico a bassa energia non perturbativo. Il processo complessivo è:
    \begin{equation*}
        p + p \to h + X
    \end{equation*}
    dove l'adrone di riferimento $h$ è dato dal decadimento del partone $c$ proveniente dallo scattering $a + b$ dei partoni del protone $a + b \to c + d$.

    Possiamo ora esprimere la sezione d'urto invariante della produzione dell'adrone nel medio range di rapidità per collisioni $pp$ come:
    \begin{equation*}
        \dv{\sigma_{pp}^h}{y d^2 p_{T}} = K \sum_{abcd}{\int{\dd{x_a} \dd{x_b} f_{a}(x_a,Q^2) f_{b}(x_b,Q^2) \dv{\hat{t}} \sigma(a + b \to c + d) \frac{D_{{h}/{c}}^0}{\pi z_c}}}
    \end{equation*}
    con:
    \begin{description}
        \item[$f_i(x_i,Q^2)$] \textit{Funzioni di Distribuzione dei Partoni} (PDF) riferite ai partoni del protone.

        \item[$\dv{\hat{t}} \sigma(a + b \to c + d)$] sezione d'urto elementare perturbativa QCD della produzione della particella $c$ a partire dallo scattering dei partoni $a + b$.

        \item[$D_{{h}/{c}}^0$] \textit{Funzione di Frammentazione} (FF): elemento adimensionale che fornisce la probabilità che il partone $c$ adronizzi nell'adrone finale $h$ emettendo gluoni e trasportando una frazione del momento del partone iniziale. Le FF non sono calcolabili perturbativamente e devono essere quindi misurate sperimentalmente.
    \end{description}

    \subsection{Funzioni di Distribuzione di Partoni (PDF)}
    \label{sec:PDF}
        Le evidenze sperimentali hanno mostrato che gli adroni non sono particelle elementari puntiformi, ma sono composte da \textit{partoni}: quark e gluoni. Come detto nella sezione~\ref{sec:QPM}, i costituenti interni di un adrone possono essere divisi in \textit{quark di valenza}, ossia i quark che effettivamente determinano i numeri quantici dell'adrone come $uud$ per il protone $p$, e in partoni del mare o \textit{mare}, ossia tutti i restanti partoni, gluoni e quark, creati e distrutti nei processi virtuali che avvengono all'interno dell'adrone secondo la QCD.

        Consideriamo l'esempio del protone $p$. Denotiamo con
        \begin{description}
            \item[$q^v(x)$] la densità di probabilità di un quark di valenza,

            \item[$q^s(x)$] la densità di probabilità di un quark del mare,

            \item[$g(x)$] la densità di probabilità di un gluone,

            \item[$x$] la frazione del momento totale trasportato da un quark $q$ o un gluone $g$.
        \end{description}
        Sapendo che i quark di valenza del protone sono $uud$, otteniamo la condizione:
        \begin{equation*}
            \int_{0}^{1}{\dd{x} u^v(x)} = 2 \qquad \int_{0}^{1}{\dd{x} d^v(x)} = 1.
        \end{equation*}
        I quark del mare sono sempre prodotti in coppie $q\bar{q}$ e danno un contributo nullo al numero barionico:
        \begin{equation*}
            \int_{0}^{1}{\dd{x} [u^s(x) - \bar{u}^s(x)]} = 0 \qquad \int_{0}^{1}{\dd{x} [d^s(x) - \bar{d}^s(x)]} = 0.
        \end{equation*}
        La stessa condizione è valida per gli altri quark del mare $s^s$, $c^s$, $b^s$ e $t^s$.
        Il momento totale portato da tutti i partoni deve contribuire al momento totale, perciò si ha la condizione:
        \begin{equation*}
            \int_{0}^{1}{\dd{x} x [u^{v}(x) + d^{v}(x) + \sum_{q} (q^{s}(x) + \bar{q}^{s}(x))]} = 1
        \end{equation*}
        I quark pesanti sono inclusi nella presente trattazione, ma sono attivi solamente se la scala di energia $Q$ del sistema è superiore alla massa $m_q$ del quark pesante stesso.

        In figura~\ref{fig:6-pdf-momentum-proton} sono mostrate le funzioni di distribuzione del momento dei partoni del protone a $Q^2 =$ \qty{10}{\giga \eV^2}. È interessante notare come il termine dei gluoni rappresenti circa metà del momento totale.

        \begin{figure}[h]
            \centering
            \includegraphics[width=0.7\linewidth]{res/fig/1-chapter/6-pdf-momentum-proton.jpg}
            \caption{Funzioni di distribuzione del momento dei partoni del protone $xf(x)$, secondo la parametrizzazione CTEQ6M dei partoni a $Q^2 =$ \qty{10}{\giga \eV^2}. La distribuzione dei gluoni è divisa per 10 per migliorarne la visualizzazione~\cite{Fusconi_2022}.}
            \label{fig:6-pdf-momentum-proton}
        \end{figure}

\newpage

    \subsection{Funzioni di Frammentazione (FF)}
        Per comprendere meglio le Funzioni di Frammentazione consideriamo il processo di annichilazione di un sistema elettrone-positrone per produrre una coppia quark-antiquark~\cite{Vogt_2007}
        \begin{equation*}
            e^{+} e^{-} \to q \bar{q}.
        \end{equation*}
        Se l'energia della collisione è $Q$ allora l'energia del fascio è $E_f = Q/2$, in maniera simmetrica, e i quark prodotti hanno energia $E_q = E_f$. Dunque se l'adrone $h$ dello stato finale ha energia $E_h$, questo porterà una frazione di energia data da
        \begin{equation*}
            z = \frac{E_h}{E_q} = \frac{2 E_h}{Q}.
        \end{equation*}
        La sezione d'urto differenziale per la produzione di adroni come funzione di $z$ è:
        \begin{equation*}
            \dv{\sigma(e^{+} e^{-} \to h X)}{z} = \sum_{q}{\sigma(e^{+}e^{-} \to q \bar{q}) \, [D_{q}^{h}(z) + D_{\bar{q}}^{h}(z)]}.
        \end{equation*}
        Questa formula è data dall'applicazione del teorema di fattorizzazione senza le PDF e giustificata dal fatto che gli elettroni sono particelle elementari. La Funzione di Frammentazione $D_{q}^{h}(z)$ rappresenta la probabilità che l'adrone $h$ dello stato finale trasporti una frazione $z$ del momento iniziale del quark, descrive quindi la transizione partone-adrone nello stesso modo in cui la PDF descrive la struttura partonica di un adrone. Per quanto detto la somma delle energie di tutti gli adroni prodotti deve formare l'energia del quark iniziale:
        \begin{equation*}
            \sum_{h}{\int_{0}^{1}{\dd{z} z \, D_{q}^{h}(z)}} = \sum_{h}{ \int_{0}^{1}{\dd{z} z \, D_{\bar{q}}^{h}(z)}} = 1.
        \end{equation*}
        La molteplicità di $h$ è data dalla somma delle probabilità di produrre $h$ da tutti i possibili quark e antiquark:
        \begin{equation*}
            n_{h} = \sum_{q}{\int_{z_{\min}}^{1} \dd{z} [D_{q}^{h}(z) + D_{\bar{q}}^{h}(z)]}
        \end{equation*}
        dove $z_{\min} = 2 m_{h} / Q$ è l'energia di soglia necessaria per produrre un adrone di massa $m_{h}$.

        Le FF possono avere diverse parametrizzazioni. Spesso è utilizzata quella in cui
        \begin{equation*}
            D_{q}^{h}(z) = N \, \frac{(1-z)^n}{z}
        \end{equation*}
        con $N$ e $n$ costanti specifiche per un dato adrone $h$. I parametri sono ottenuti sperimentalmente dal fit dell'immensa molte di dati disponibile per collisioni $e^{+} e^{-}$.

        Si ipotizza che le Funzioni di Frammentazione siano universali, pertanto una volta calcolati i parametri per le collisioni $e^{+} e^{-}$, questi dovrebbero essere applicabili in altri casi come le collisioni $ep$, $pp$ e $p\bar{p}$.

\newpage

\section{Adronizzazione di sapori pesanti in collisioni A-A}
    Fin dalle prime osservazioni di produzione di adroni in collisioni tra ioni pesanti fu evidente che il processo di \textit{adronizzazione} fosse più complesso della sola \textit{frammentazione} nel vuoto. I modelli che tentano di spiegare questa differenza considerano che avvenga in concomitanza anche un secondo meccanismo chiamato \textit{ricombinazione} o \textit{coalescenza}. La differenza tra i due processi è che nella
    \begin{description}
        \item[frammentazione] il momento iniziale è distribuito tra i frammenti, mentre nella

        \item[ricombinazione] due o tre partoni vicini nello spazio delle fasi (posizione e momento) producono un adrone con momento trasverso pari alla somma dei momenti dei partoni iniziali
    \end{description}
    come mostrato il figura~\ref{fig:7-fragm-recom}.

    Il calcolo degli effetti di ricombinazione nelle collisioni tra ioni pesanti è particolarmente complesso poiché non è possibile scrivere la funzione d'onda di tutti i partoni che costituiscono il QGP.

    La probabilità di trovare due o tre partoni vicini nello spazio delle fasi diminuisce all'aumentare del momento trasverso dell'adrone nello stato finale, per questo la ricombinazione contribuisce meno ad \textit{alti impulsi trasversi} $p_T$ dove la frammentazione risulta il fenomeno dominante. Inoltre l'effetto della ricombinazione è più significativo \textit{in collisioni centrali} poiché queste ultime favoriscono maggiormente la transizione a QGP, mentre la frammentazione è tipica delle \textit{collisioni periferiche}.

    \begin{figure}[h]
        \centering
        \includegraphics[width=0.61\linewidth]{res/fig/1-chapter/7-fragm-recom.jpg}
        \caption{I meccanismi di ricombinazione e frammentazione in atto per creare lo stesso adrone nello stato finale in funzione dell'impulso trasverso $p_{T}$~\cite{Vogt_2007}.}
        \label{fig:7-fragm-recom}
    \end{figure}

\newpage

\section{Rapporto di produzione barione/mesone}
\label{sec:BARIONE/MESONE}
    Come detto nella sezione~\ref{sec:ADRONIZATIONpp}, i quark pesanti sono di grande interesse per lo studio delle proprietà del QGP poiché, creati in coppia solo nei primissimi istanti della collisione, attraversano il sistema durante tutte le fasi della sua evoluzione interagendo coi suoi costituenti e fornendo una misura diretta delle sue proprietà.

    Alcuni modelli teorici prevedono una \textit{produzione di barioni}, stati legati di 3 quark, \textit{più abbondante} di quella di mesoni, stati legati di 2 quark, in un mezzo denso deconfinato (QGP) per effetto di processi di adronizzazione per \textit{ricombinazione} (coalescenza) tra quark che si aggiungono al processo di adronizzazione per \textit{frammentazione}. Lo studio del \textit{rapporto di produzione barioni/mesoni} in collisioni A-A e $pp$ è quindi un importante strumento per studiare l'effetto del QGP sull'adronizzazione dei quark.

    $\Lambda_{c}^{+}(udc)$ e $D^{0}(c \bar{u})$ sono rispettivamente il barione e il mesone più leggeri contenenti un quark charm e possono essere identificati in un ampio intervallo di momento, per questo si prestano molto bene per valutare il rapporto di produzione barione/mesone nelle diverse collisioni.

    \subsection{Adroni charmati in collisioni \ch{Pb}-\ch{Pb} a $\sqrt{s_{NN}} =$ \qty{5.02}{\tera \eV}}
        Contrariamente a quanto atteso, i valori misurati del rapporto barione/mesone mostrati in figura~\ref{fig:8-ALICE-pp13TeV-pp5.02TeV-pPb5.02TeV-PbPb5.02TeV} \textit{non} differiscono in maniera significativa tra di loro e in particolare \textit{non} si osserva il consistente aumento della produzione di barioni charmati, riferito alla produzione di mesoni charmati, previsto dall'insorgere di meccanismi di \textit{ricombinazione} in collisioni A-A (qui \ch{Pb}-\ch{Pb}). Infatti, nei processi di \textit{ricombinazione} o coalescenza, la formazione di un barione è molto \textit{meno} sfavorita rispetto alla formazione di un mesone a differenza dei processi di \textit{frammentazione} che invece la disincentivano. Questo porterebbe a \textit{prevedere} un valore del rapporto barione/mesone \textit{maggiore} in collisioni A-A~\cite{Strazzi_2019} come accennato sopra, sezione~\ref{sec:BARIONE/MESONE}. Una possibile spiegazione è che tali o simili meccanismi siano \textit{già} presenti e importanti, soprattutto a basso $p_T$, \textit{anche} in collisioni $pp$ e $p$-A alle energie di LHC. L'andamento del rapporto in funzione della molteplicità dimostra che, se presenti, tali meccanismi sono \textit{già} all’opera anche a basse molteplicità.

        Con l'espressione \textit{molteplicità} o \textit{classi di molteplicità} intendiamo il numero di particelle secondarie prodotte in un evento. Questo termine aiuta a organizzare e analizzare i dati delle collisioni: gli eventi con alta molteplicità possono essere più complessi, ma potenzialmente più informativi rispetto a quelli a bassa molteplicità. Questa categorizzazione permette una migliore comprensione dei processi fisici coinvolti.

\newpage
        
        \begin{figure}[t]
            \centering
            \includegraphics[width=0.83\linewidth]{res/fig/1-chapter/8-ALICE-pp13TeV-pp5.02TeV-pPb5.02TeV-PbPb5.02TeV.pdf}
            \caption{Valore del rapporto barione/mesone $\Lambda_{c}^{+}/D^{0}$ misurato dall'esperimento ALICE a LHC integrato su tutto lo spettro degli impulsi trasversi $p_{T}$ del barione $\Lambda_{c}^{+}$ (sono stati utilizzati valori estrapolati dove non erano presenti misure sperimentali) per diversi sistemi collidenti: protone-protone $pp$, protone-nucleo $p$-\ch{Pb} e nucleo-nucleo \ch{Pb}-\ch{Pb}, sia centrali sia periferici, in funzione della molteplicità~\cite{Kalteyer_ALICE_2022}. Si può notare come il rapporto sia praticamente compatibile per tutti i tipi di collisioni entro gli errori sperimentali.}
            \label{fig:8-ALICE-pp13TeV-pp5.02TeV-pPb5.02TeV-PbPb5.02TeV}
        \end{figure}

    \subsection%
    [Adroni charmati in collisioni $pp$ a $\sqrt{s} =$ \qty{5.02}{\tera \eV} e a $\sqrt{s} =$ \qty{13}{\tera \eV}]%
    {Adroni charmati in collisioni $pp$ a $\sqrt{s} =$ \qty{5.02}{\tera \eV} e a \\ $\sqrt{s} =$ \qty{13}{\tera \eV}}
        I valori del rapporto di produzione $\Lambda_{c}^{+}/D^{0}$ misurati in collisioni $pp$ alle energie di LHC risultano significativamente \textit{maggiori} rispetto a quanto misurato in collisioni $ep$ e $e^{+} e^{-}$ e soprattutto rispetto a modelli teorici che assumono \textit{solo} processi di frammentazione e utilizzano funzioni di frammentazione (FF) basate su tali esperimenti. Questi modelli prevedono un valore del rapporto di circa \num{0.1}, con una debole dipendenza dal valore dell'impulso trasverso, significativamente inferiore al valore compreso tra \num{0.4} e \num{0.6} misurato in ALICE~\cite{ALICE_2018_pp7Tev_pPb5.02TeV}~\cite{ALICE_2021_pp5.02TeV_pPb5.02TeV_prod}~\cite{ALICE_2021_pp5.02TeV_pPb5.02TeV_prod_ratio}~\cite{ALICE_2022_pp13TeV} a bassi lavori di $p_{T}$ (figura~\ref{fig:9-ALICE-pp5.02TeV-pp7TeV-pp13TeV}). Questa discrepanza può essere interpretata come una indicazione del fatto che le probabilità che un quark charm adronizzi in uno specifico adrone charmato, ovvero le Funzioni di Frammentazione (FF), \textit{non} siano universali come ritenuto fino ad ora, ma dipendano dalle caratteristiche del sistema collidente.
        
        Il rapporto $\Lambda_{c}^{+}/D^{0}$ in funzione dell'impulso trasverso sembra inoltre variare se considerato in diverse \textit{classi di molteplicità} (figura~\ref{fig:8-ALICE-pp13TeV-pp5.02TeV-pPb5.02TeV-PbPb5.02TeV}), con il risultato, per collisioni $pp$ ad elevata molteplicità, che si avvicina molto a quanto ottenuto in collisioni \ch{Pb}-\ch{Pb} ad energie del centro di massa nucleone-nucleone di \qty{5.02}{\tera \eV}.

        I rapporti di produzione misurati da ALICE nel range di rapidità centrali $\abs{\eta} <$ \num{0.9} in collisioni $pp$ a diverse energie del centro di massa riportati in figura~\ref{fig:9-ALICE-pp5.02TeV-pp7TeV-pp13TeV} mostrano un certo \textit{accordo} per i valori di impulso \textit{fuori} dal bin $p_{T} =$ \qtyrange[range-phrase = {,}, range-units = single, per-mode = symbol, range-units = bracket, range-open-bracket = [, range-close-bracket = ]]{0}{1}{\giga \eV \per \clight}, con un andamento descrescente, mentre i valori al suo \textit{interno}, calcolati unicamente attraverso il canale di decadimento $\Lambda_{c}^{+} \to p K_{S}^{0}$, sono molto diversi. I risultati espressi in intervalli di $p_{T}$ sono compatibili entro gli errori sperimentali. In effetti l'analisi dei decadimenti a basso impulso trasverso è particolarmente delicata.

        \begin{figure}[h]
            \centering
            \includegraphics[width=0.83\linewidth]{res/fig/1-chapter/9-ALICE-pp5.02TeV-pp7TeV-pp13TeV.pdf}
            \caption{Rapporto di produzione degli adroni charmati $\Lambda_{c}^{+}$ e $D^{0}$ in funzione dell'impulso trasverso $p_{T}$ in collisioni $pp$ a energia cinetica nel centro di massa di $\sqrt{s} =$ \qty{5.02}{\tera \eV}, $\sqrt{s} =$ \qty{7}{\tera \eV} e $\sqrt{s} =$ \qty{13}{\tera \eV} misurati col rivelatore ALICE a LHC~\cite{ALICE_2023_pp5.02TeV_pp7TeV_pp13TeV}.}
            \label{fig:9-ALICE-pp5.02TeV-pp7TeV-pp13TeV}
        \end{figure}

        Le misure sperimentali prodotte dall'esperimento ALICE sono state confrontate con diversi modelli teorici e fenomenologici, al fine di verificarne l'attendibilità e la capacità di riprodurre i dati sperimentali. I modelli teorici riportati in figura~\ref{fig:10-ALICE-pp13TeV} sono:
        \begin{description}
            \item[PYTHIA 8.243 Monash 2013]~\cite{SCR_2014} un generatore Monte Carlo (MC) che implementa \textit{solo} processi di \textit{frammentazione} con FF per adroni charmati basate sulle misurazioni ottenute con collisioni $e^{+} e^{-}$. Predice un rapporto $\Lambda_{c}^{+}/D^{0}$ di circa \num{0.1} con una debole dipendenza da $p_{T}$ e costituisce una grossa sottostima dei dati sperimentali, soprattutto a bassi range di impulso trasverso.

            \item[PYTHIA 8.243]~\cite{CS_2015} un generatore MC che implementa la \textit{riconnessione di colore}. Questo modello di adronizzazione è basato sul \textit{modello a stringhe di Lund}. Sono presenti tre possibili modalità che introducono vincoli più o meno restrittivi sulla generazione: Mode 0 senza vincoli, Mode 2 con vincoli stretti, Mode 3 con vincoli più larghi. Questo modello predice \textit{piuttosto bene} l’andamento del rapporto $\Lambda_{c}^{+}/D^{0}$ in particolare nella Mode 0.

            \item[SHM+RQM] (Statistical Hadronization Model - Relativistic Quark Model)~\cite{MR_2019} un modello che calcola le frazioni di adroni charmati basandosi su \textit{densità termiche}, dunque dipendenti dalla massa dello stato e dal fattore di degenerazione di spin. Fa uso di ulteriori stati barionici eccitati ancora non misurati, ma che si assume esistano secondo il modello relativistico dei quark (RQM). Secondo tale modello, l'incremento nella produzione di barioni charmati non è dovuto a $\Lambda_{c}^{+}$ primarie, ma a decadimenti di stati barionici di massa più elevata ($\Lambda_{c}^{+}$ di feed-down). Le previsioni di questo modello sono buone per tutti i range di $p_{T}$.

            \item[Catania]~\cite{PMDCG_2018} un modello che assume che anche in collisioni $pp$ si possa creare uno stato di QGP e che dunque l’adronizzazione avvenga sia per \textit{frammentazione} che per \textit{ricombinazione} (coalescenza).
            
            \item[QCM]~\cite{SLS_2018_QCM} è un modello che ipotizza la formazione di adroni charmati, a basso $p_{T}$, dalla combinazione di quark charm con quark più leggeri ($u$, $d$, $s$) che si muovono alla stessa velocità.

            \item[POWLANG]~\cite{BDPPMN_2023_POWLANG} similmente al modello Catania, assume la creazione di uno stato di QGP anche in collisioni $pp$ ed utilizza lo stesso meccanismo di adronizzazione in-medium sviluppato per descrivere i risultati ottenuti con collisioni Pb-Pb. In questo modello, la formazione di barioni charmati avviene dalla combinazione di quark charm con stati di diquark leggeri eccitati presenti nel plasma. Tra i vari modelli proposti, questo è quello che riproduce in maniera meno accurata l'andamento del rapporto $\Lambda_{c}^{+}/D^{0}$ in funzione di $p_{T}$.
        \end{description}
        Questi modelli, escluso PYTHIA 8.243 Monash 2013, forniscono previsioni simili in quasi tutto il range di impulso trasverso $p_{T}$, tranne nel bin \qtyrange[range-phrase = {,}, range-units = single, per-mode = symbol, range-units = bracket, range-open-bracket = [, range-close-bracket = ]]{0}{1}{\giga \eV \per \clight}. L’\textit{analisi dati in questo range} è dunque molto importante, perché permette di \textit{distinguere i modelli teorici} più affidabili da quelli che lo sono meno. È tuttavia un’analisi molto difficile a causa del bassissimo rapporto segnale su fondo. Inoltre, come si può vedere in figura~\ref{fig:10-ALICE-pp13TeV}, l’errore statistico e quello sistematico sono significativi, il che rende la misura sperimentale meno attendibile e di difficile interpretazione.

        \begin{figure}[p]
            \centering
            \includegraphics[width=0.83\linewidth]{res/fig/1-chapter/10-ALICE-pp13TeV.pdf}
            \caption{Rapporto di produzione $\Lambda_{c}^{+}/D^{0}$ in funzione dell'impulso trasverso $p_{T}$ in collisioni $pp$ a $\sqrt{s} =$ \qty{13}{\tera \eV} confrontato con diversi modelli teorici~\cite{ALICE_2023_pp13TeV}.}
            \label{fig:10-ALICE-pp13TeV}
        \end{figure}