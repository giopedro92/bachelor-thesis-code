%-----------------------------INTRODUZIONE
A Large Ion Collider Experiment (ALICE) è l’esperimento al Large Hadron Collider (LHC) del CERN dedicato allo studio delle collisioni tra ioni pesanti ultrarelativistici. Il programma fisico dell’esperimento prevede anche studi con ioni leggeri e con collisioni protone-protone e protone-ione come riferimento per le misure effettuate con collisioni tra ioni pesanti. Le recenti analisi sull’adronizzazione dei quark pesanti in collisioni $pp$ hanno dimostrato risultati sorprendenti, compatibili con fenomeni di ricombinazione (coalescenza) o con la creazione di uno stato di Quark-Gluon Plasma (QGP), non attesi per tali sistemi collidenti. Lo studio del barione charmato $\Lambda^{+}_{c}$ è di particolare interesse in questo campo. La misura della sua sezione d’urto di produzione a bassi impulsi traversi $p_{T}$ infatti permette di valutare il grado di precisione con cui i diversi modelli teorici e fenomenologici, sviluppati negli ultimi anni in seguito ai risultati di ALICE, sono in grado di riprodurre le misure sperimentali.

La ricostruzione di tale particella con metodi di analisi standard, che applicano tagli rettangolari su alcuni parametri caratteristici dei suoi decadimenti, risulta estremamente complessa a causa della sua breve vita media e dell’elevato fondo combinatoriale. Negli ultimi anni quindi le analisi condotte all’interno della collaborazione ALICE hanno fatto largo uso di tecniche di Machine Learning (ML) sempre più sofisticate. 

Nel presente lavoro di tesi si è iniziata la scrittura di un framework di analisi per la ricostruzione del barione $\Lambda_{c}^{+}$ attraverso l’utilizzo di reti neurali convoluzionali (CNN). Sebbene strumenti di questo tipo esistano già sul mercato, come ad esempio la suite TMVA distribuita all’interno del software ROOT, la realizzazione di un framework totalmente indipendente offre notevoli vantaggi come una maggiore flessibilità e la possibilità di aggiungere nuove features e di personalizzarlo ed ampliarlo a piacimento. Per l’implementazione software sono state utilizzate le librerie open source TensorFlow e l’API Keras. Il pacchetto, comprendente al momento solo la parte di pre-analisi sulle variabili di ingresso e di addestramento della rete neurale, è stato testato con dati raccolti dall’esperimento ALICE riguardanti il barione $\Lambda_{c}^{+}$ ed in particolare il suo decadimento adronico $\Lambda_{c}^{+} \to p K^{0}_{S}$. L’analisi si è concentrata nell’intervallo di impulso trasverso $1 < pT < 2$ \unit{\giga \eV \per \clight}. 

\begin{comment}
Nel presente lavoro di tesi il barione $\Lambda^{+}_{c}$ è stato ricostruito attraverso il suo decadimento $\Lambda_{c}^{+} \to p K^{0}_{S}$ utilizzando i dati raccolti dall’esperimento ALICE in collisioni $pp$ ad una energia del centro di massa di $\sqrt{s} =$ \qty{13}{TeV}. L’analisi si è concentrata nell’intervallo di impulso trasverso $1 < p_{T} < 2$ \unit{\giga \eV \per \clight} poichè a basso $p_{T}$ i vari modelli teorici mostrano una certa discrepanza nelle loro previsioni ed è quindi possibile attraverso l’analisi dei dati sperimentali valutare l’attendibilità di tali previsioni così come fornire input per correzioni ai vari modelli. Data la difficoltà di questa misura, sono state utilizzate tecniche di Machine Learning per mezzo dell'allenamento di una Rete Neurale. In particolare sono state utilizzate le librerie open source TensorFlow e l’API Keras creando un pacchetto indipendente da quelli già esistenti, come TMVA di ROOT, personalizzabile e ampliabile a piacimento.
\end{comment}

Il capitolo~\ref{cha:1-QCD} presenta una breve introduzione alla fisica del Modello  Standard e in particolare alla teoria della Cromodinamica Quantistica (QCD) approfondendo il Quark Gluon Plasma (QGP) e il processo di adronizzazione dei quark pesanti nei diversi sistemi collidenti evidenziando l’importanza della misura della sezione d’urto di produzione della particella $\Lambda^{+}_{c}$.

Nel capitolo~\ref{cha:2-ALICE} è brevemente introdotto l’esperimento ALICE del CERN e in particolare i suoi rivelatori le cui informazioni sono state utilizzate nell’analisi presentata in questa tesi.

Nel capitolo~\ref{cha:3-LAMBDA+c} si accenna brevemente al Machine Learning, alle Neural Network e al loro impiego nell'analisi qui svolta. Sono mostrati i risultati dell'addestramento della rete e i differenti metodi di valutazione della bontà dello stesso.