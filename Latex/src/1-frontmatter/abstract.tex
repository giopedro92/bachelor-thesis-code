%---------------------------------ABSTRACT
L’esperimento ALICE (A Large Ion Collider Experiment) a LHC (Large Hadron Collider) presso il CERN di Ginevra è dedicato allo studio delle collisioni tra ioni pesanti ultrarelativistici. Il programma di ricerca dell’esperimento però prevedere anche studi con ioni leggeri e con collisioni protone-protone e protone-ione al fine di confrontare le misure ottenute nelle collisioni tra ioni pesanti. L’obiettivo principale dell’esperimento è quello di studiare lo stato della materia chiamato Quark Gluon Plasma (QGP) che si forma in condizioni di altissima temperatura e densità di energia. A causa della sua breve vita media però, lo studio del QGP può essere condotto solo tramite misure indirette. Uno degli strumenti migliori per l’analisi delle sue proprietà è lo studio dei quark pesanti: questi, grazie alle loro masse elevate, vengono prodotti nelle primissime fasi della collisione, si propagano all’interno del QGP e interagiscono col sistema durante tutte le fasi della sua evoluzione. Le più recenti analisi sull’adronizzazione dei quark pesanti in collisioni protone-protone hanno mostrato però risultati inaspettati, compatibili con fenomeni di ricombinazione (coalescenza) o con la formazione di uno stato di QGP, non attesi in tali collisioni. Per interpretare correttamente tali risultati, il barione charmato $\Lambda^{+}_{c}$ è di particolare interesse; è possibile infatti valutare la correttezza dei diversi modelli teorici e fenomenologici, sviluppati in questi anni in seguito ai risultati di ALICE, misurando la sua sezione d’urto di produzione a bassi impulsi trasversi. La ricostruzione di questa particella risulta però complessa a causa della sua breve vita media e del basso rapporto tra segnale e fondo; per questo motivo è stato implementato un programma di analisi basato su tecniche di Machine Learning (ML) e di Neural Network (NN) per poter insegnare al modello come separare segnale e fondo nei dati. Questo programma, realizzato con il linguaggio Python, rappresenta il primo passo verso la realizzazione di un framework indipendente, nuovo rispetto a quelli utilizzati nelle analisi High Energy Physics (HEP). L’addestramento delle reti si è rivelato essere funzionante e ha mostrato la propria correttezza testato su dati prodotti dall’esperimento ALICE in cui si è considerato il canale di decadimento adronico $\Lambda_{c}^{+} \to p K^{0}_{S}$.

\begin{comment}
L'esperimento ALICE (A Large Ion Collider Experiment) a LHC (Large Hadron Collider) presso il CERN di Ginevra è dedicato allo studio delle collisioni tra ioni pesanti ultrarelativistici. Il programma di ricerca dell'esperimento però prevedere anche studi con ioni leggeri e con collisioni protone-protone e protone-ione al fine di confrontare le misure ottenute nelle collisioni tra ioni pesanti. L'obiettivo principale dell'esperimento è quello di studiare lo stato della materia chiamato Quark Gluon Plasma~(QGP) che si forma in condizioni di altissima temperatura e densità di energia. Le più recenti analisi sull'adronizzazione dei quark pesanti in collisioni protone-protone mostrano risultati inaspettati, compatibili con fenomeni di ricombinazione (coalescenza) o con la formazione di uno stato di QGP, \textit{non} attesi in collisioni $pp$. Il QGP però è difficilmente osservabile a causa della sua breve vita media per cui le prove della sua esistenza sono tutte dovute a misure indirette. Uno degli strumenti migliori per l'analisi delle sue proprietà è lo studio dei quark pesanti: questi, grazie alle loro masse elevate, vengono prodotti nelle primissime fasi della collisione, si propagano all'interno del QGP e interagiscono col sistema durante tutte le fasi della sua evoluzione. In questo ambito lo studio del barione charmato $\Lambda^{+}_{c}$ è di particolare interesse infatti è possibile valutare la correttezza dei diversi modelli teorici e fenomenologici, sviluppati in questi anni in seguito ai risultati di ALICE, misurando la sezione d'urto di produzione a bassi impulsi trasversi. Nella presente tesi è stato considerato il canale di decadimento $\Lambda_{c}^{+} \to p K^{0}_{S}$. La ricostruzione di questa particella risulta però complessa a causa della sua breve vita media e del basso rapporto tra segnale e fondo, per questo motivo è stato fatto uso di tecniche di Machine Learning~(ML) per mezzo di una Neural Network~(NN) per poter ``insegnare'' alla rete a separare segnale e fondo nei dati. A tale scopo è stato creato col linguaggio \texttt{Python} un framework indipendente, nuovo rispetto a quelli utilizzati nelle analisi High Energy Physics~(HEP), che si è rivelato essere funzionante e ha mostrato la propria correttezza nell'allenamento della rete neurale sui dati simulati di segnale e fondo.
\end{comment}